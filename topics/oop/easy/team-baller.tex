\begin{blocksection}
\question Write \lstinline$TeamBaller$, a subclass of \lstinline$Baller$. An instance of \lstinline$TeamBaller$ cheers on the team every time it passes a ball.

\ifprintanswers\else
\begin{lstlisting}
class TeamBaller(_______________):
    """
    >>> jamie = BallHog('Jamie')
    >>> cheerballer = TeamBaller('Ethan', has_ball=True)
    >>> cheerballer.pass_ball(jamie)
    Yay!
    True
    >>> cheerballer.pass_ball(jamie)
    I don't have the ball
    False
    """
    def pass_ball(_______________, ________________):
\end{lstlisting}
\fi

\begin{solution}[1in]
\begin{lstlisting}
class TeamBaller(Baller):
    """
    >>> jamie = BallHog('Jamie')
    >>> cheerballer = TeamBaller('Ethan', has_ball=True)
    >>> cheerballer.pass_ball(jamie)
    Yay!
    True
    >>> cheerballer.pass_ball(jamie)
    I don't have the ball
    False
    """
    def pass_ball(self, other):
        did_pass = Baller.pass_ball(self, other)
        if did_pass:
            print('Yay!')
        else:
            print("I don't have the ball")
        return did_pass
\end{lstlisting}
\end{solution}

\begin{guide}
\textbf{Teaching Tips}
\begin{itemize}
  % defining should have the same functionality as in the Baller class, but with slight modification.
  \item Remember that \lstinline{pass_ball} that we're defining should have the same functionality as in the Baller class, but with slight modification.
  \item Ask students what should happen if the ball doesn't get passed (this should hint to them to somehow use the boolean retyrn type of \lstinline{pass_ball})
\end{itemize}
\end{guide}
\end{blocksection}
