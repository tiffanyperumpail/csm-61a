\twocolumn
\begin{blocksection}

\question \textbf{Flying the cOOP} What would Python display? \\
Write the result of executing the code and the prompts below. If a function is returned, write "Function". If nothing is returned, write "Nothing". If an error occurs, write "Error".

\vspace{2\baselineskip}

\begin{lstlisting}
class Bird:
    def __init__(self, call):
        self.call = call
        self.can_fly = True
    def fly(self):
        if self.can_fly:
            return "Don't stop me now!"
        else:
            return "Ground control to Major Tom..."
    def speak(self):
        print(self.call)

class Chicken(Bird):
    def speak(self, other):
        Bird.speak(self)
        other.speak()

class Penguin(Bird):
    can_fly = False
    def speak(self):
        call = "Ice to meet you"
        print(call)

andre = Chicken("cluck")
gunter = Penguin("noot")
\end{lstlisting}
\end{blocksection}

\newpage
\begin{blocksection}
\vspace{9\baselineskip}
\begin{lstlisting}
>>> andre.speak(Bird("coo"))
\end{lstlisting}
\begin{solution}[.2in]
cluck \\
coo
\end{solution}

\vspace{3\baselineskip}
\begin{lstlisting}
>>> andre.speak()
\end{lstlisting}
\begin{solution}[.2in]
Error
\end{solution}

\vspace{3\baselineskip}
\begin{lstlisting}
>>> gunter.fly()
\end{lstlisting}
\begin{solution}[.2in]
"Don't stop me now!"
\end{solution}

\vspace{3\baselineskip}
\begin{lstlisting}
>>> andre.speak(gunter)
\end{lstlisting}
\begin{solution}[.2in]
cluck \\
Ice to meet you
\end{solution}

\vspace{3\baselineskip}
\begin{lstlisting}
>>> Bird.speak(gunter)
\end{lstlisting}
\begin{solution}[.2in]
noot
\end{solution}
\end{blocksection}
\onecolumn

\begin{blocksection}
\begin{guide}
\textbf{Teaching Tips}
\begin{itemize}
\item Consider giving a mini lecture on inheritance before doing this question
\item It may be helpful to use a \href{https://goo.gl/KkFr9F}{live environment diagram of the execution}
\item Emphasize the order of variable lookup
\begin{itemize}
  \item When looking for a variable, go from instance to class to parent until failure
\end{itemize}
\item Emphasize the different ways to pass in \lstinline{self} for any function, either using dot notation or passing it in as a parameter
\begin{itemize}
  \item For \lstinline{Bird.speak(gunter)}, \lstinline{Bird} is not passed in for \lstinline{self}, since \lstinline{Bird} is a class, not an instance
\end{itemize}
\item For \lstinline{andre.speak(Bird("coo"))}, an "unnamed" \lstinline{Bird} object is created and passed in as a parameter
\begin{itemize}
  \item Make sure students understand how this works- it can help to draw a temporary box for this \lstinline{Bird} object in your OOP diagram and erase it afterwards
\end{itemize}
\end{itemize}
\end{guide}
\end{blocksection}
