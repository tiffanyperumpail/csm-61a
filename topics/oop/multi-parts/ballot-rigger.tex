\begin{blocksection}
\question
\textbf{a. Ballot Rigger} In the 2020 CSM election, Jade's campaign advisors have decided to push back against the election fraud by rigging ballots themselves. Let's see how many ballots they were able to rig!

Fill in the \texttt{Ballot} Class such that each ballot's \texttt{vote} attribute corresponds to the name of the person they vote for.

In the ballot, bubble 'a' corresponds to 'Jade', 'b' corresponds to 'Jason', 'c' corresponds to 'Richard', and 'd' corresponds to 'Other'. Fill out the blanks above to categorize each voter's choice.

\emph{Hint:} It may be useful to let \texttt{choices} be a dictionary of bubble choices to names.

\begin{lstlisting}
class Ballot:
    choices = __________________________ 
    def __init__(self, name, bubble):
        self.name = name
        self.vote = _________________________

california = [Ballot('kenny', 'c'), Ballot('alina', 'd'), Ballot('jamie', 'b')]
white_house = [Ballot('jade', 'a'), Ballot('jason', 'a'), Ballot('catherine', 'a')]
postman = []
postman.extend(white_house)
postman.append(california)
\end{lstlisting}

\begin{solution}[1in]
\begin{lstlisting}
class Ballot:
    choices = {'a': 'Jade','b':'Jason', 'c':'Richard', 'd':'Other'}
    def __init__(self, name, bubble):
        self.name = name
        self.vote = Ballot.choices[bubble]
california = [Ballot('kenny', 'c'), Ballot('alina', 'd'), Ballot('jamie', 'b')]
white_house = [Ballot('jade', 'a'), Ballot('jason', 'a'), Ballot('catherine', 'a')]
postman = []
postman.extend(white_house)
postman.append(california)
\end{lstlisting}
\end{solution}
\end{blocksection}

\begin{blocksection}
\vspace{1\baselineskip}
\textbf{b. Changing Ballots}
The Ballot Rigger takes all ballots and rigs them, changing votes for Jason and Other to Jade. Implement this in the lines of code above.
\vspace{1\baselineskip}
\begin{lstlisting}
class BallotRigger:
    def rig(self, ballot):
        ___________________________________
        	_______________________________

bad_guy = __________________
for ballot in california:
    bad_guy.__________________
for ballot in white_house:
    bad_guy.__________________


\end{lstlisting}
\begin{solution}[1in]
\begin{lstlisting}
class BallotRigger:
    def rig(self, ballot):
        if ballot.vote == 'Jason' or ballot.vote == 'Other':
            ballot.vote = 'Jade'
bad_guy = BallotRigger()
for ballot in california:
    bad_guy.rig(ballot)
for ballot in white_house:
    bad_guy.rig(ballot)
\end{lstlisting}
\end{solution}
\end{blocksection}

\begin{blocksection}
\vspace{1\baselineskip}
\textbf{c. Tallying Votes}
After the Ballot Rigger has rigged the ballots, return the number of votes for each candidate, implement the \lstinline{count_votes} method, which adds votes to the count dictionary. 
\emph{Hint:} Our ballot list can contain both Ballots and lists of Ballots! How can we deal with nested lists of ballots?
\vspace{1\baselineskip}
\begin{lstlisting}
choices = {'Jade': 0 ,'Jason': 0, 'Richard': 0, 'Other': 0}
def count_votes(ballots):
    for b in ballots:
        if isinstance(________,________):
            _______________
        else:
            ____________ += 1

ballot_counter = postman[:]
count_ballots(ballot_counter)


\end{lstlisting}
\begin{solution}[1in]
\begin{lstlisting}
choices = {'Jade': 0 ,'Jason': 0, 'Richard': 0, 'Other': 0}
def count_votes(ballots):
    for b in ballots:
        if isinstance(b, list):
            count_votes(b)
        else:
            choices[b.vote] += 1

ballot_counter = postman[:]
count_votes(ballot_counter)
\end{lstlisting}
\end{solution}
\end{blocksection}



\begin{blocksection}
\vspace{1\baselineskip}
\textbf{d. Final Count}
Fill in the final tallies of votes for each candidate.
\emph{Hint:} It might help to draw box and pointer diagrams for \lstinline{postman} abd \lstinline{ballot_counter}.
\vspace{1\baselineskip}
\begin{lstlisting}
choices = {'Jade': __ ,'Jason': __, 'Richard': __, 'Other': __}

\end{lstlisting}
\begin{solution}[1in]
\begin{lstlisting}
choices = {'Jade': 5 ,'Jason': 0, 'Richard': 1, 'Other': 0}
\end{lstlisting}
\end{solution}

\end{blocksection}
