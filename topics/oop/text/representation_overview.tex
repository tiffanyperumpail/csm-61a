\textbf{Representation Overview: \lstinline{__repr__} and \lstinline{__str__}}

The goal of \lstinline{__str__} is to convert an object to a human-readable string. The \lstinline{__str__} function is helpful for printing objects and giving us information that's more readable than \lstinline{__repr__}. Whenever we call \lstinline{print()} on an object, it will call the \lstinline{__str__} method of that object and print whatever value the \lstinline{__str__} call returned. For example, if we had a \lstinline{Person} class with a name instance variable, we can create a \lstinline{__str__} method like this:
\begin{lstlisting}
def __str__(self):
    return "Hello, my name is " + self.name
\end{lstlisting}
This \lstinline{__str__} method gives us readable information: the person's name. Now, when we call print on a person, the following will happen:
\begin{lstlisting}
>>> p = Person("John Denero")
>>> str(p)
'Hello, my name is John Denero'
>>> print(p)
Hello, my name is John Denero
\end{lstlisting}

The \lstinline{__repr__} magic method of objects returns the "official" string representation of an object. You can invoke it directly by calling \lstinline{repr(<some object>)}. However, \lstinline{__repr__} doesn't always return something that is easily readable, that is what \lstinline{__str__} is for. Rather, \lstinline{__repr__} ensures that all information about the object is present in the representation. When you ask Python to represent an object in the Python interpreter, it will automatically call \lstinline{repr} on that object and then print out the string that \lstinline{repr} returns. If we were to continue our \lstinline{Person} example from above, let's say that we added a \lstinline{repr} method:

\begin{lstlisting}
def __repr__(self):
    return "Name: " + self.name
\end{lstlisting}

Then we can write the following code:

\begin{lstlisting}
# Python calls this object's repr function to see what
# to print on the line. Note, Python prints whatever
# result it gets from repr so it removes the quotes
# from the string
>>> p
Name: John Denero

# User is invoking the repr function directly.
# Since the function returns a string, its output
# has quotes. In the previous line, Python called
# repr and then printed the value. This line works
# like a regular function call: if a function
# returns a string, output that string with quotes.
>>> repr(p)
"Name: John Denero"
\end{lstlisting}

\begin{blocksection}
\begin{guide}
\textbf{Teaching Tips}
\begin{itemize}
\item \lstinline{__repr__} and \lstinline{__str__} error if they do not return strings
\item Writing the correct quotation marks for \lstinline{str} and \lstinline{repr} results is required on midterms and can be confusing for students
\begin{itemize}
  \item \lstinline{string} leaves quotation marks
  \begin{itemize}
    \item \lstinline{"test" = 'test'}
  \end{itemize}
  \item \lstinline{print} removes quotation marks
  \begin{itemize}
    \item \lstinline{print("test") = test}
  \end{itemize}
  \item \lstinline{repr} adds quotation marks
  \begin{itemize}
    \item \lstinline{repr("test") = "'test'"}
  \end{itemize}
  \item \lstinline{str} leaves quotation marks
  \begin{itemize}
    \item \lstinline{str("test") = 'test'}
  \end{itemize}
  \item \lstinline{repr} and \lstinline{str} both add quotation marks to numbers or booleans
  \begin{itemize}
    \item \lstinline{repr(3) = '3'}
  \end{itemize}
\end{itemize}
\end{itemize}
\end{guide}
\end{blocksection}
